%% Adapted from work by Christophe Roger (Darwiin)
% ------------------------------------------------------------------------- %
\documentclass[cv]{cv_style} 
\usepackage[left=1cm,top=1cm,right=1cm,bottom=1cm,nohead,nofoot]{geometry}
% ------------------------------------------------------------------------- %
\begin{document}

% ------------------------------------------------------------------------- %
%                                  TOP
% ------------------------------------------------------------------------- %
{
\adjustboxset{width = \textwidth, keepaspectratio = true, valign = t}
\begin{tabular}{p{\topbarleftsize\textwidth} > {\arraybackslash}p{\topbarrightsize\textwidth}}
 \vspace{1cm}\\
    \vspace{-.2cm}
    \begin{minipage}[t]{\topbarleftsize\textwidth}
        % \begin{center}
        \vspace{-.1cm} \\
        \begin{minipage}{\textwidth}
            %\begin{flushright}
                \Large
                \textbf{\color{blue2}Canditure au poste de Data Scientist\\
            }\\
                %\topdraw  \\
                \vspace{-0.8cm} \\
                {\textbf{Pierre \hspace{.1cm}Gauthier}}
            %\end{flushright}
        \end{minipage}
        % \end{center}
    \end{minipage}

    & % --------------------------------------------------------------- % 

    \begin{minipage}[t]{\topbarrightsize\textwidth}
        \vspace{0.5cm}
        \begin{tikzpicture}[remember picture,overlay]
            \node[
                text width=10cm
                %, anchor=west
                , right
            ] at (-3.94,-0.4999) {        %%%%%%%%%%%% Classique
                %] at (-2.96,-0.4999) {        %%%%%%%%%%%% Ingénieur R&D
                \begin{minipage}{\textwidth}
                        \Large
                        %\textbf{\color{blue2}Ingenieur R{\&}D\\
                        % \textbf{\color{blue2}Data Engineer\\
                         %\textbf{\color{blue2}Data Analyst \\
                             %\textbf{\color{blue2}consultant Data Analytics\\
                        % \textbf{\color{blue2}Data Scientist\\
                    % }\\
                \end{minipage}
            };  
        \end{tikzpicture} 
        \vspace{-0.4cm}\\
        %\Topbox{CONTACT}{-2.54}           %%%%%%%%%%%% Classique
        \Topbox{CONTACT}{-9.92}      %%%%%%%%%%%% Ingénieur R&D
				\vspace{0.1cm} \\
        % \socialinfo{
        {\color{white}h} \hspace{1.80cm}    \email{pierrezachariasgauthier@gmail.com} \\
            \vspace{.3cm}
         {\color{white}h}\hspace{1.80cm}   \smartphone{(+33) 6.78.15.86.37} \\
        % }
    \end{minipage}\\
\end{tabular}
}
{
% ------------------------------------------------------------------------- %
\adjustboxset{width = \textwidth, keepaspectratio = true, valign = t}
% ------------------------------------------------------------------------- %
\vspace{0.0cm}\\
\begin{tabular}{p{\leftsize\textwidth} > {\arraybackslash}p{\rightsize\textwidth}}
    \textsize
    \begin{minipage}[t]{\leftsize\textwidth}

% ------------------------------------------------------------------------- %
%                                  SIDEBAR
% ------------------------------------------------------------------------- %

        % \begin{textblock}{40}(6,30)
        \small
        \vspace{-0.85cm}
        \Infobox{LIENS}{2.4}{1}
						\vspace{-0.50cm}\\
            % \begin{minipage}{4cm}
            \begin{flushright}
            \hfill \githubSymbol \hspace{0.5em} \href{http://www.github.com/pierrzacharias}{git/pierrzacharias} \\
						\vspace{.1cm}\\
            \hfill \linkedinSymbol \hspace{0.5em} \href{https://www.linkedin.com/in/pierre-zacharias-gauthier-293750133/}{in/pierre-gauthier} \\
            \end{flushright}
            % \end{minipage}

        \vspace{0.5cm}\\

        \Infobox{PRESENTATION}{0.1}{0}
        \vspace{-0.50cm}\\
        \begin{justify}   
					Actuellement data scientist au CEA. Je suis ouvert à des propositions pour des projets en science des données. Disponible à partir de mi-Mai 2021. Autonome et Analytique.  \\
        \end{justify}
        % \sidebarsection{ Info}

        \vspace{0.2cm}\\

				\Infobox{INFORMATIONS}{0.05}{0}
        \vspace{-0.50cm}\\
        \begin{flushright}
            6 rue Montesquieu\\
						\vspace{.1cm}\\
            Grenoble, France \\
						\vspace{.1cm}\\
            24 ans, Français  \\
						\vspace{.1cm}\\
            Permis B
        \end{flushright}

        \vspace{0.6cm}\\
        
        
        \Infobox{LANGUES}{1.48}{1}
            \vspace{-0.3cm}\\
            \begin{flushright}
                % \begin{minipage}{2cm}
                    
                    %\begin{tabular}{m{1.5cm}{2cm}l}
                          Français\\
						\vspace{.1cm}\\
                          Anglais (B2)
                    %\end{tabular}
                % \end{minipage}
            \end{flushright}

        \vspace{0.5cm} 

        \Infobox{LOISIRS}{1.95}{1}
        \begin{flushright}
            \vspace{-0.35cm}\\
            \small
            Lecture\\ 
						\vspace{.1cm}\\
						Natation \\
						\vspace{.1cm}\\
						 Musique (basse)\\ 
        \end{flushright}
            
    % \end{textblock}
% ------------------------------------------------------------------------- %
    \end{minipage}
&
    \hspace{0.4cm}
    \begin{minipage}[t]{\rightsize\textwidth}
    %\mainbartextsize    
%--------------------------------------------------------------------------------------------
    
% ------------------------------------------------------------------------- %
%                                  EXPERIENCES
% ------------------------------------------------------------------------- %

    \titlebox{EXPÉRIENCES PROFESIONNELLES}{8.45}{13.75}
    \vspace{-0.2cm}\\
    
    \frcventrycurrent
				{En poste - \hspace{0.6cm}}
				{Sep 2019   \hspace{0.6cm}}
				{Data Scientist \hfill  [ 7 mois ] }
        {CEA}
        {Grenoble, France}
        {
					Travaux au sein du laboratoire LETI dans le cadre du laboratoire commun avec l'entreprise \href{https://www.diabeloop.fr/}{Diabeloop} sur le diabète de type 2. \\
					\vspace{0.0cm} 
          \rightchevron\hspace{.1cm} Travail de biostatiscien sur une étude clinique avec le nettoyage des données médicales, la mise en place d'une base de données, la réalisation de représentations graphiques, la recherche de métriques pertinentes, l'analyse statistique. Présentations et discussions des résultats avec différents acteurs. Participation aux activités de soumission et de publication.\\
					\vspace{0.0cm} 
          \rightchevron\hspace{.1cm} Travail sur le développement d'un simulateur python pour la simulation physiologique de patients diabetiques de type 2 à partir de publications dans le domaine.
        }
        {Utilisés :}
				{Python (pandas, sklearn, scipy), SQL, git, Excel}

    \vspace{0.6cm}\\

    \frcventry
        {2019 \hspace{0.6cm}}   
        {Data Scientist / Data Engineer \hfill  [ 6 mois ] }
        {OCTOPEEK}
        {Enghien-les-bains, France}
        {
					Stage de fin d'étude. Développement de plusieurs outils pour l'enrichissement de données. \\
            \vspace{0.0cm}
            \rightchevron\hspace{.1cm} Travail sur la partie Data d'une application de prospection géographique pour des \\
						profesionnels. Nettoyage, mise en forme et stockage des données dans SQL et Elaticsearch. 
						Jointure de données de différentes sources à l'aide d'une distance intertextuelle avec des hyperparamètres optimisés.
						Implémentation des fonctionnalités géographiques SQL.\\
            \vspace{0.0cm}
						\rightchevron\hspace{.1cm} Développement d'un outil d'extraction de contenus web modulable (développement avec les librairies python Selenium et Beautiful Soup).  Mise en place de stratégies pour résister à la \\
						détection de robots web et aux changements d'architectures entre les pages web. L'outil vise ainsi à palier aux mises à jours des sites et aux protections par les changements d'architecture. Déploiement de l'outil pour une utilisation en interne sous forme de micro-services avec une scalabilité à l'aide de Docker, RabbitMQ et Traefik. Gestion des logs avec logstash.\\
            \vspace{0.0cm}
            \rightchevron\hspace{.1cm} Utilisation d'un modèle de gradient-boosting pour le calcul d'une probabilité qu'un \\professionel soit retraité à partir des informations obtenues sur son profil.\\
            \vspace{0.0cm}
            \rightchevron\hspace{.1cm} Utilisation de l'outil de scraping développé pour enrichir les données d'un client. \\Nettoyage des données fournies, adaptation de l'outil pour ce besoin spécifique, puis mise en production. Mise en forme des données en sortie pour les livrables.
        }
				% {Utilisés {\hspace{-0.06cm}:}}
				{Utilisés :}
				{Python(pandas, sklearn, selenium, beautifulsoup), SQL, Elasticsearch, git, \\RabbitMQ, Docker, Traefik}

    \vspace{0.6cm}\\


			\frcventry
					{2018\hspace{0.6cm}} 
					{Stage en Developpement Web \hfill [ 2 mois ]}
					{MICHELANGE}
					{Arnouville, France}
					{
						 Développement web principalement front-end. Travail sur le CSS de plusieurs sites, développement de fonctionnalités en PHP. 
					}
					{Utilisés :}
					{HTML, CSS, PHP}

% ------------------------------------------------------------------------- %

\end{minipage}\\
\end{tabular}

\vspace{0.3cm}\\
\begin{tabular}{p{16.5cm} > {\arraybackslash}p}
\\& page 1/3\\
\end{tabular}

% ------------------------------------------------------------------------- %
\newpage

\hspace{0.4cm}
\begin{minipage}[t]{\textwidth}

% ------------------------------------------------------------------------- %
%                                  EDUCATION
% ------------------------------------------------------------------------- %

\vspace{1.2cm}\\

        \titleboxbis{FORMATION}{3.2}{18.18}
        \vspace{.2cm}\\
        \begin{liste}
            \educationentryfr
                {2016-2019}
                {Ingénieur Civil des Mines }
                {
				École d'ingénieur généraliste.\\ 
				Département Génie Industriel et Mathématiques Appliquées (GIMA).\\
				(Probabilités et statistiques, analyse de données, machine learning, séries \\temporelles, mathématiques financières, équations différentielles, théorie de \\ l'information). 
				Spécialisé en science des données et en programmation.
                }
                { École Nationale Supérieure des Mines de Nancy, France }
		  \vspace{-.6cm}\\
		  \educationentryfr
                {2014-2016}
                {Classes préparatoires aux grandes écoles }
                { Filière PCSI / PSI* (Mathématiques, Physique, Informatique, Sciences de l'Ingénieur). }
                { Lycée Marcelin Berthelot, Saint-Maur-Des-Fossés, France }
                    
		  \vspace{-.6cm}\\
		  \educationentryfr
			 {2014}
                {Baccalauréat}
                { Baccalauréat Scientifique.  }
                { Lycée Saint Charles à Athis-Mons, France }
        \end{liste}
% ------------------------------------------------------------------------- %
%                                  SKILL
% ------------------------------------------------------------------------- %

\vspace{0.3cm}\\
			\titlebox{COMPÉTENCES TECHNIQUES}{7.33}{18.18}
        %\renewcommand{\arraystretch}{1.1}

        \begin{skillenv}
            % \rightchevron{}
					Connaissances en mathématiques appliquées, en science des données et en développement \\informatique ainsi qu'une expérience sur des projets scientifiques divers.\\ 
            
						\vspace{-0.1cm}\\

            \skilltitlefr{Apprentissage - Data Analysis}{-0.5}

								\vspace{-0.1cm}\\
								\skilllist{
										& Fondamentaux en apprentissage automatique, analyse de données et statistiques. \\
										&	Séries temporelles. Algèbre, Optimisation mathématique. Connaissances de base \\ 
										& en apprentissage profond avec l'analyse sémantique du langage.
								}
								\vspace{.2cm}\\
								\skilllevelfr
									{Python (\textit{sklearn, scipy, ...}), Matlab}
									{R (\textit{glmnet, forecast, ggplot2, bsts})}                
									{Pytorch} \\                           

						\vspace{0.1cm}\\

            \skilltitlefr{Data Processing}{-0.5}

								\vspace{-0.1cm}\\

								\skilllist{
										& Bases de données SQL et noSQL.  Connaissance de la suite  Elasticsearch avec \\
										& Logstah, Kibana. Traitement des données avec Python. Déploiement de services \\ 
										&	avec Docker, RabbitMQ et Traefik. Dashboard avec seaborn.
								}
								\vspace{.2cm}\\
								\skilllevelfr
									{Python (pandas, matplotlib, seaborn), SQL}
									{Elasticsearch, Docker, RabbitMQ, Traefik}                
									{Logstash, Kibana} \\                           

						\vspace{0.1cm}\\

            \skilltitlefr{Software Engineering}{-0.5}

						\vspace{-0.2cm}\\

            \skilllevelfr
								{Git}
								{C++, HTML/CSS, Unix}
								{Javascript, Scala, Bash}

            \vspace{0.4cm}\\

            \skilltitlefr{Outils}{-0.105}

            \vspace{-0.7cm}\\

            \skilllist{
							& Vim, \LaTeX, Markdown, Adobe Suit, Excel, VBA
						}

        \end{skillenv}


		\vspace{0.2cm}\\
        

% ------------------------------------------------------------------------- %
\end{minipage}\\
\end{tabular}

\vspace{0.56cm}\\
\begin{tabular}{p{16.5cm} > {\arraybackslash}p}
\\& page 2/3\\
\end{tabular}

% ------------------------------------------------------------------------- %
\newpage

\hspace{0.4cm}
\begin{minipage}[t]{\textwidth}
\vspace{0.5cm}\\

% ------------------------------------------------------------------------- %
%                                  PROJECT
% ------------------------------------------------------------------------- %

%\hphantom\\

        \titleboxbis{PORTFOLIO}{3.08}{18.0}
        % \vspace{-0.1cm}\\
        \begin{minipage}[t]{.94\textwidth}
    
        \vspace{-0.3cm}\\

        \cvproject
        {Apprentissage automatique en physique quantique}
        {Inria Grand-Est, France}
        {Prédiction de l'énergie d'atomisation de molécules organiques pour la conception de panneaux solaires.\\
            \rightchevron\hspace{.1cm} Calcul des descripteurs chimiques des molécules. \\
            \rightchevron\hspace{.1cm} Utilisation d'un modèle de régression à noyaux.
        }
        {
            \faGithub: \link{https://github.com/pierrzacharias/Projet_3A}{Quantum Mechanics in a Nutshell} 
        }
        {Utilisé :}
        {Python (\textit{sklearn, pandas, numpy})} 
\vspace{-0.45cm}\\
        \cvproject
        {Modélisation d'un écoulement chaotique}
        {Inria Grand-Est, France}
        {Projet de recherche pour modéliser un écoulement chaotique par une simulation numérique en prenant en compte le caractère non-homogène des champs magnétiques qui génèrent l'écoulement.
        }
        {
            \faGithub: \link{https://github.com/pierrzacharias/Projet2A_eliecartan}{Mélange par un écoulement stationnaire de fluide à faible Reynolds} 
					% \vspace{-1.1cm}
        }
        {Utilisé :}
        {Matlab}
        \cvproject
        {\textbf{Sorry ARIMA, but I’m Going Bayesian}}
        {}
        {Confrontation pour la prédiction des séries temporelles de l'approche fréquentiste usuelle (modèle SARIMA) avec l'approche bayésienne par la méthode bsts de Monte-Carlo par chaînes de Markov.  
        }
        {
            \faGithub: \link{https://github.com/pierrzacharias/time_series_project}{Sorry ARIMA} 
        }
        {Utilisé :}
        {R (\textit{glmnet, bsts, ggplot2})}

        \cvproject
        {EDP start up challenge}
        {}
        {Réalisation d'un dashboard dynamique avec seaborn dans le cadre d'un \link{https://www.iberdrola.com/innovation/international-startup-program-perseo/collaborative-solutions-for-electric-car-recharges}{Startup Challenge Européen}. Le but du projet était de proposer une statégie pour le déploiement de nouvelle bornes de chargement pour voitures électriques sur le territoire espagnol. 
        }
        {
            \faGithub: \link{https://github.com/Ouardavalue/EDP}{EDP charging points} 
        }
        {Utilisé :}
        {Python (pandas, seaborn)}

        \end{minipage}
        \end{minipage}



\vspace{7.95cm}\\
\begin{tabular}{p{16.5cm} > {\arraybackslash}p}
\\& page 3/3\\
\end{tabular}
% ------------------------------------------------------------------------- %
    \end{minipage}
\end{tabular}


}
% ------------------------------------------------------------------------- %
% \begin{tabular}{p{\leftsize\textwidth} > {\arraybackslash}p{\rightsize\textwidth}}

%     \begin{minipage}[t]{\leftsize\textwidth}
%         \vspace{-0.016cm} 

	% \invisible{NLP, Linux}
	% \invisible{Data Mining}
	% \invisible{IA, machine learning}
	% \invisible{word embedding}
	% \invisible{Big Data}
	% \invisible{Analytics}
	% \invisible{Data visualisation pipeline}
	% \invisible{Data visualization}
	% \invisible{R programming}
	% \invisible{Deep learning}
	% \invisible{Hadoop}
	% \invisible{Predictive Modeling}
	% \invisible{noSQL}
	% \invisible{Database}
	% \invisible{Data Engineer Engineering }
	% \invisible{ETL}
	% \invisible{Predictive Analytics}
	% \invisible{Batch processing}
	% \invisible{Data lake}
	% \invisible{MapReduce}
	% \invisible{Apache Spark}
	% \invisible{Stream processing}
	% \invisible{decision tree}
	% \invisible{AI}
	% \invisible{project management}
	% \invisible{neural networks}
	% \invisible{data engineering}
	% \invisible{analysis}
	% \invisible{Amazom AWS, Tableau}
	% \invisible{Scala}
	% \invisible{Airflow}



    % \end{minipage}    



%\input{section_experience_short}		% Section experiences
%\input{section_project}


%\input{section_experiences}				% Section experiences 
%	\input{section_publications}			% Section Publications
%	\input{section_awards}					% Section Awards
%\input{section_references} 				% Section References
     

% ------------------------------------------------------------------------- %
%                                  END
% ------------------------------------------------------------------------- %

\end{document}
